\documentclass[a4paper, 12pt]{article}
\usepackage{color}
\usepackage{scalerel}
\usepackage{tikz}
\usepackage{geometry}

\geometry{
		total = {160mm, 237mm},
		left = 25mm,
		right = 35mm,
		top = 30mm,
		bottom = 30mm,
	}

\usepackage{tabularx}
\usepackage{fancyhdr}
\usepackage{graphicx}
\usepackage{amssymb}
\usepackage{amsmath}
\usepackage{multicol}
\usepackage{graphicx}
\usepackage{wrapfig}
\usepackage{enumitem}
\usepackage{lastpage}
\usepackage{transparent}
\usepackage{cancel}
\usepackage{setspace}
\onehalfspacing

\newcommand{\ans}{\textbf{Jawab}:}
\newcommand{\R}{\mathbb{R}}
\newcommand{\N}{\mathbb{N}}

\begin{document}
\pagenumbering{gobble}
    \begin{tabular}{|lcl|}
     \hline
     Nama&:&Dhanar Agastya Rakalangi\\
     NRP&:&5002221075\\
     \hline
    \end{tabular}

    \begin{enumerate}
        \item Perhatikan barisan fungsi $(f_n)$ yang didefinisikan dengan $f_n(x)=\dfrac{nx}{1+nx^2}$ untuk $x\in A:=[0,\infty)$.
        \begin{enumerate}
            \item Tunjukkan bahwa $(f_n)$ terbatas pada $A$ untuk semua $n\in\N$.\\
            \ans \\
            Perhatikan $f_n(x)=\dfrac{nx}{1+nx^2}$ dengan $x \in A$, maka $x > 0 \ , nx \geq 0$ dan $1+nx^2 \geq 1$ . Selanjutnya dengan definisi, terdapat $M>0$ sedemikian hingga $|f_n| \leq M$, maka $f_n(x) \leq \dfrac{nx}{1}$ .
            Dengan demikian, $f_n(x)$ terbatas pada $A$ untuk semua $n\in\N$.

             \item Tunjukkan bahwa $(f_n)$ konvergen titik-demi-titik ke suatu fungsi $f$, tetapi tidak terbatas.\\
             \ans
            \begin{itemize}
                \item Untuk $x=0$, kita punya $f_n(0)=0$ untuk setiap $n\in\N$. Sehingga $f_n(x)$ konvergen ke $0$.
                \item Untuk $x>0$, kita punya $f_n(x)=\dfrac{nx}{1+nx^2}=\dfrac{1}{1/nx+x}\implies\dfrac{1}{x}$. 
                Sehingga $f_n(x)$ konvergen ke $1/x$.
            \end{itemize}
            Jadi, dapat disimpulkan $(f_n)$ konvergen titik-demi-titik ke suatu fungsi $f$ yaitu 
            $$f(x)=\begin{cases}0&\text{jika }x=0\\1/x&\text{jika }x>0\end{cases}$$.\\
            Selanjutnya akan ditunjukkan bahwa $f$ tidak terbatas, kita gunakan kontradiksi. Asumsikan $f$ terbatas, maka ada $M>0$ sehingga $|f(x)|\leq M$ untuk setiap $x\in A$. Kita ambil $x=\dfrac{1}{(3M)}$, maka $f\left(\dfrac{1}{(3M)}\right)=3M$ , jelas ini kontradiksi dengan asumsi bahwa $f$ terbatas.\\\\
            $\therefore$ $f$ tidak terbatas.\\

            \item Apakah $(f_n)$ konvergen seragam pada $A$? Jelaskan!\\
            \ans\\
            Tidak konvergen seragam, karena $f$ tidak kontinu pada $A$, padahal $(f_n)$ kontinu untuk setiap $n\in\N$.
        \end{enumerate}

        \item Jika $\sum a_n$ konvergen mutlak dan $(b_n)$ barisan terbatas, tunjukkan bahwa $\sum a_nb_n$ konvergen mutlak.\\
        \ans\\
        Karena $\sum a_n$ konvergen mutlak, maka $\sum |a_n|$ konvergen. Karena $(b_n)$ terbatas,
        maka ada $M>0$ sehingga $|b_n|\leq M$ untuk setiap $n\in\N$. Dengan demikian, kita punya 
        $|a_nb_n|\leq M|a_n|$ untuk setiap $n\in\N$. Karena $\sum |a_n|$ konvergen, maka $\sum M|a_n|$
        juga konvergen. Dengan demikian, $\sum a_nb_n$ konvergen mutlak.

        \item Tunjukkan bahwa deret $\dfrac{1}{1^2}+\dfrac{1}{2^3}+\dfrac{1}{3^2}+\dfrac{1}{4^3}+\dots$ adalah konvergen,tetapi uji rasio dan uji akar gagal \\~\\
        diterapkan untuk memeriksa konvergensi deret tersebut.\\

        \ans\\
        Kita perhatikan bahwa deret tersebut dapat ditulis sebagai berikut
        \begin{align*}
            \sum_{n=1}^{\infty}\dfrac{1}{(2n-1)^2}+\dfrac{1}{(2n)^3}=\sum_{n=1}^{\infty}\dfrac{1}{(2n-1)^2}+\sum_{n=1}^{\infty}\dfrac{1}{(2n)^3}
        \end{align*}

        \begin{itemize}
            \item Uji Rasio : 
            \begin{align*}
                 \sum_{n=1}^{\infty}\dfrac{1}{(2n-1)^2} &= \lim_{n \to \infty} \dfrac{f_{n+1}}{f_n} \\
                 &= \lim_{n \to \infty} \dfrac{\dfrac{1}{(2n)^2}}{\dfrac{1}{(2n-1)^2}} = \lim_{n \to \infty} \dfrac{(2n-1)^2}{2n^2} = 1 \ (\text{Tidak dapat ditentukan}) \\
                 \sum_{n=1}^{\infty}\dfrac{1}{(2n)^3} &= \lim_{n \to \infty} \dfrac{f_{n+1}}{f_n} \\
                 &= \lim_{n \to \infty} \dfrac{\dfrac{1}{(2n+1)^3}}{\dfrac{1}{2n}^3} = 1 \ (\text{Tidak dapat ditentukan})
            \end{align*}

            \newpage
            \item Uji Akar : 
            \begin{align*}
                 \sum_{n=1}^{\infty}\dfrac{1}{(2n-1)^2} &= \lim_{n \to \infty} |x_n|^{\frac{1}{n}}\\
                 &= \lim_{n \to \infty} \left|\dfrac{1}{(2n-1)^{\frac{2}{n}}}\right| = e ^{(\frac{2}{n}) \ln (2n-1)} = e^0 = 1 \ (\text{Tidak dapat ditentukan}) \\
                 \sum_{n=1}^{\infty}\dfrac{1}{(2n)^3} &= \lim_{n \to \infty} |x_n|^{\frac{1}{n}}\\
                 &= \lim_{n \to \infty} \left| \dfrac{1}{(2n)^{\frac{3}{n}}}\right| = e ^{(\frac{3}{n}) \ln (2n)} = e^0 = 1 \ (\text{Tidak dapat ditentukan}) \\
            \end{align*}
        \end{itemize} 

        Kita perhatikan bahwa deret $\sum_{n=1}^{\infty}\dfrac{1}{(2n-1)^2}$ adalah deret $p$-harmonik dengan $p=2>1$ yang konvergen. 
        Demikian pula dengan deret $\sum_{n=1}^{\infty}\dfrac{1}{(2n)^3}$ adalah deret $p$-harmonik dengan $p=3>1$ yang konvergen juga.
        Sehingga deret tersebut konvergen.

        \item Diberikan $\sum a_n$ deret yang konvergen mutlak. Tunjukkan bahwa $\sum a_n \sin(nx)$ adalah deret yang konvergen mutlak dan seragam.\\
        \ans\\
        Karena $\sum a_n$ konvergen mutlak, maka $\sum |a_n|$ konvergen. Karena $\sin(nx)$ terbatas sehingga $|\sin(nx)|\leq 1$ untuk setiap $n\in\N$.
        Dengan demikian, kita punya $|a_n\sin(nx)|\leq |a_n|$ untuk setiap $n\in\N$. Sehingga didapatkan $\sum |a_n\sin(nx)|\leq \sum |a_n|$.
        Dengan kriteria uji banding, maka $\sum a_n\sin(nx)$ konvergen mutlak.\\
        Untuk menunjukkan bahwa $\sum a_n\sin(nx)$ konvergen seragam terutama pada interval $[0,2\pi]$, kita gunakan kriteria Weierstrass M. 
        Dalam kasus ini, kita dapat mengambil \(f_n(x) = a_n \sin(nx)\) dan \(M_n = |a_n|\). Kita sudah tahu bahwa \(\sum |a_n|\) konvergen, 
        maka sesuai Kriteria Weierstrass M, \(\sum a_n \sin(nx)\) konvergen seragam pada interval \(x \in [0, 2\pi]\).\\

    \end{enumerate}

\end{document}